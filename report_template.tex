\documentclass[12pt,a4paper]{article}
\usepackage[utf8]{inputenc}
\usepackage{geometry}
\usepackage{graphicx} % For diagrams
\usepackage{hyperref} % For clickable references
\usepackage{amsmath}  % For mathematical notation
\usepackage{float}    % For figure placement
\usepackage{enumitem} % For better lists
\geometry{margin=1in}

\title{ 
    \vspace{0.3em} 
    CMPE 300 - Analysis of Algorithms \\
    Project 2 Report \\
}
\author{
    Name Surname 1 \\ 
    Student Number 1 \\ 
    \and
    Name Surname 2 \\ 
    Student Number 2
}


\begin{document}

\maketitle

\tableofcontents
\newpage

\section{Introduction}
\label{sec:intro}
Provide a brief overview of the project. 

---

\section{Design Decisions and Assumptions}
\label{sec:design}
Describe the major design decisions you made while implementing the project. Include any assumptions that were necessary to simplify or clarify the implementation.

\textbf{Questions to Address:}
\begin{itemize}
    \item How did you structure the grid and the units?
    \item What partitioning strategy did you use, and why?
    \item How did you handle unit movements, attacks, and healing?
\end{itemize}

---

\section{Implementation Details}
\label{sec:implementation}
Detail the steps taken to implement the project, the challenges faced, and how they were overcome.

---

\section{Partitioning Strategy}
\label{sec:partitioning}
\subsection{Strategy Used}
Specify which partitioning strategy was used (Striped or Checkered) and justify your choice.

\subsection{Communication Between Processes}
Explain how data is communicated between processes (e.g., boundary communication in striped/checkered partitioning).

\subsection{Advantages and Disadvantages}
Discuss the benefits and potential downsides of your chosen strategy in terms of workload balance, communication cost, and ease of implementation.

---

\section{Test Results}
\label{sec:results}
Provide the results of your tests. Include examples of initial and final grid states and discuss performance metrics such as execution time and communication overhead.

\subsection{Example Input and Output}
Include an example of the input and the corresponding output for a test case.

\textbf{Input:}
\begin{verbatim}
N W T R
...
\end{verbatim}

\textbf{Output:}
\begin{verbatim}
...
\end{verbatim}

\subsection{Performance Analysis}
Discuss the performance of your implementation. Include any benchmarks or comparisons made.

---

\section{Conclusion}
\label{sec:conclusion}
Summarize the overall experience of working on this project. Discuss lessons learned, potential improvements, and the benefits of using MPI for this type of simulation.

---

\section*{Appendices}
---

\section*{References}
List any references or resources you used while completing the project.

\begin{itemize}
    \item MPI Documentation: \url{https://mpi-forum.org}
    \item Python mpi4py Documentation: \url{https://mpi4py.readthedocs.io}
\end{itemize}

\end{document}
